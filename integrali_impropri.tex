\section{Integrali impropri}
\subsection{Integrazione secondo Riemann}

Data $f:[a,b] \rightarrow \mathbb{R}$ limitata, con $a=x_0 < x_1 < \ldots < x_n =b$ e $[a,b]$ intevallo limitato:
\begin{gather*}
	S(f) =\sum_{i=1}^{n} \left( \sup_{[x_{i-1},x_i]} f(x) \right) \cdot (x_i-x_{i-1}) \qquad \text{somma superiore}
	\\
	s(f) =\sum_{i=1}^{n} \left(\inf_{[x_{i-1},x_i]} f(x)\right)\cdot (x_i-x_{i-1}) \qquad \text{somma inferiore}
\end{gather*}
Se $\inf_{a=x_0 < x_1 < ... < x_n =b} S(f) = \sup_{a=x_0 < x_1 < ... < x_n =b} s(f) =I \in \mathbb{R}$ allora $f$ si dice Riemann integrabile in $[a,b]$ e si pone 
\begin{equation*}
	\int_{a}^{b} f(x) \mathrm{d}x=I.
\end{equation*}


\subsection{Integrali impropri}
\begin{exbar}
\begin{example}
	\begin{equation*}
		f(x)=\frac{1}{x^\alpha}, \qquad x \geq 1, \; \alpha \in \mathbb{R}
	\end{equation*}
	
	Vogliamo definire, se possibile, 
	\begin{equation*}
		\int_{1}^{+\infty} f(x) \ \mathrm{d}x= \int_{1}^{+\infty}\frac{1}{x^\alpha} \ \mathrm{d}x
	\end{equation*}
	Stiamo provando ad integrare una funzione limitata su un intervallo illimitato.
	\begin{center}
		\includegraphics[width=\linewidth]{integrali_impropri/pag77}
		\label{fig:pag77}
	\end{center}

	Fissato $c > 1$, riesco a definire $\int_{1}^{c} f(x)dx$ perché integro una funzione limitata su un intervallo limitato. Posso dunque definire
	\begin{align*}
		\int_{1}^{+\infty} f(x) \ \mathrm{d}x 
		&= \lim_{c\rightarrow +\infty} \int_{1}^{c}f(x) \ \mathrm{d}x \qquad \text{se esiste}
		\\
		\int_{1}^{+\infty} \frac{1}{x^\alpha} dx 
		&= \lim_{c \rightarrow + \infty} \int_{1}^{c} \frac{1}{x^\alpha} \ \mathrm{d} x 
		\\
		&=
		\begin{cases}
			\lim_{c \rightarrow +\infty}\frac{1}{1-\alpha}[c^{1-\alpha}-1] & \text{se } \alpha \neq 1
			\\[1em]
			\lim_{c \rightarrow +\infty} \ln c & \text{se } \alpha=1
		\end{cases}
		\\
		&=
		\begin{cases}
			 \frac{1}{\alpha-1} & \text{se } \alpha > 1
			\\[1em]
			+\infty & \text{se } \alpha \leq 1
		\end{cases}
	\end{align*}
\end{example}
\end{exbar}

	
\begin{exbar}
\begin{example}
	\begin{equation*}
		f(x)=-\ln x, \qquad x \in ]0,1]
	\end{equation*}
	
	e provo a definire 
	\begin{equation*}
		\int_{0}^{1} f(x) \ \mathrm{d}x
	\end{equation*}
	
	cioè l'integrale di una funzione illimitata $\left( \lim_{x \rightarrow 0^+} f(x) = + \infty \right)$ su un intervallo limitato.
	\begin{center}
		\includegraphics[width=0.75\linewidth]{integrali_impropri/pag79}
		\label{fig:pag79}
	\end{center}
	
	Riesco a definire bene $\int_{c}^{1} f(x) \ \mathrm{d}x$ perché integro una funzione limitata
	$\left( |f(x)|\leq -\ln c \quad \forall \ x \in [c,1] \right)$ su un intervallo limitato.
	
	Posso dunque definire
	\begin{align*}
		\int_{0}^{1} f(x)dx 
		&= \lim_{c\rightarrow +0^+} \int_{c}^{1} f(x) \ \mathrm{d}x \qquad \text{se esiste}
		\\
		\int_{0}^{1}(-\ln x) \ \mathrm{d}x 
		&= \lim_{c \rightarrow 0^+} \int_{c}^{1} (-\ln x) \ \mathrm{d} x
		\\
		&= \lim_{c \rightarrow 0^+} [x-x\ln x]_{c}^{1} 
		\\
		&= \lim_{c \rightarrow 0^+} [1-c+c\ln c]=1
	\end{align*}
\end{example}
\end{exbar}

\begin{definition}
	Sia $f:[a,b[ \ \rightarrow \mathbb{R}; b \in \mathbb{R} \cup\{+\infty\}$ (voglio definire $\int_{a}^{b}f(x) \ \mathrm{d}x$) Riemann integrabile in $[a,c] \ \forall c \in [a,b[$. Se $\exists$ finito il $\lim_{c \rightarrow b^-} \int_{a}^{c} f(x) \ \mathrm{d}x $, allora $f$ si dice integrale in senso improprio o generalizzato in $[a,b[$ e la quantità 
	\begin{equation*}
		\int_{a}^{b} f(x) \ \mathrm{d} x = \lim_{c\rightarrow b^-} \int_{a}^{c} f(x) \ \mathrm{d} x
	\end{equation*}
	si dice integrale improprio o generalizzato di $f$ in $[a,b[$.
	
	Analogamente, data $f:]a,b]\rightarrow \mathbb{R}; a \in \mathbb{R} \cup \{-\infty\}$, Riemann integrabile in $[c,b] \ \forall c \in ]a,b]$. Se $\exists$ finito il $\lim_{c \rightarrow a^+} \int_{c}^{b} f(x) \ \mathrm{d}x$, allora $f$ si dice integrabile in senso improprio o generalizzato in $]a,b]$ e la quantità
	\begin{equation*}
		\int_{a}^{b} f(x) \ \mathrm{d}x = \lim_{c \rightarrow a^+} \int_{c}^{b} f(x) \ \mathrm{d}x
	\end{equation*}
	si dice integrale improprio o generalizzato di $f$ in $]a,b]$.
	
	Se $f$ è integrabile in senso improprio in un intervallo $I$, allora si dice anche che l'integrale (improprio) di $f$ in $I$ è convergente o che $f$ ha integrale (improprio) convergente in $I$.
	
	Se il limite che definisce l'integrale improprio di $f$ è infinito, si dice anche che l'integrale (improprio) di $f$ è divergente o che $f$ ha integrale (improprio) divergente in $I$.
\end{definition}

\begin{attbar}
		Vista la definizione, 
	\begin{equation*}
		\int_{1}^{+\infty} \frac{1}{x^\alpha} \ \mathrm{d}x \text{ converge} \iff \alpha > 1
	\end{equation*}
\end{attbar}

\begin{exbar}
\begin{example} \textbf{importante}
	\begin{align*}
		f(x)=\frac{1}{(x-a)^\alpha}, \qquad
		& x \in ]a, a+1], 
		\\ & a \in \mathbb{R} \text{ fissato}
		\\ & \alpha \in \mathbb{R} \text{ parametro}
		\\ & \alpha > 0
	\end{align*}
	
	Studiamo la convergenza di 
	\begin{align*}
		\int_{a}^{a+1} \frac{1}{(x-a)^\alpha} \ \mathrm{d}x  
		&= \lim_{c \rightarrow a^+} \int_{c}^{a+1} \frac{1}{(x-a)^\alpha} \ \mathrm{d}x
		\\
		&=
		\begin{cases}
			\lim_{c \rightarrow a^+} \frac{1}{1-\alpha} \left[1 - (c-a)^{1-\alpha} \right] & \text{se } \alpha \neq 1 
			\\[1em]
			\lim_{c \rightarrow a^+} -\ln(c-a) & \text{se } \alpha =1
		\end{cases}\\
		&=
		\begin{cases}
			\frac{1}{1-\alpha} & \text{se } \alpha <1 
			\\[1em]
			+\infty & \text{se } \alpha \geq1.
		\end{cases}
	\end{align*}
\end{example}
\end{exbar}

\begin{attbar}
	\begin{equation*}
		\int_{a}^{a+1} \frac{1}{(x-a)^\alpha} \ \mathrm{d}x \text{ converge} \iff \alpha < 1
	\end{equation*}
	
	In particolare
	\begin{equation*}
		\int_{0}^{1} \frac{1}{x^\alpha} \mathrm{d}x \text{ converge} \iff \alpha < 1
	\end{equation*}
\end{attbar}

\begin{exbar}
\begin{example}
	Stabilire se converge 
	\begin{equation*}
		\label{eq:pag 83}
		\int_{0}^{1} \frac{\sin{\frac{1}{x}}}{x^2} \ \mathrm{d}x
	\end{equation*}
	
	\begin{center}
		\includegraphics[width=0.75\linewidth]{integrali_impropri/pag83}
		\label{fig:pag83}
	\end{center}
	
		Devo calcolare 
	\begin{equation*}
		\lim_{c\rightarrow0^+} \int_{c}^{1} \frac{\sin{\frac{1}{x}}}{x^2} \ \mathrm{d}x= \lim_{c \rightarrow 0^+} \cos{\frac{1}{x}} \bigg|_{c}^{1}=\lim_{c\rightarrow 0^+}\left(1-\cos{\frac{1}{c}}\right)
	\end{equation*}
	
	che non esiste $\Rightarrow$ la funzione $f(x)=\frac{\sin{\frac{1}{x}}}{x^2}$, non è integrabile in senso improprio in $]0,1]$	
\end{example}
\end{exbar}

\begin{exbar}
\begin{example} \textbf{importante}
	\begin{align*}
		\int_{0}^{+\infty} e^{\alpha x} \ \mathrm{d}x 
		&= \lim_{c \rightarrow +\infty} \int_{0}^{c} e^{\alpha x} \ \mathrm{d}x 
		\\ & =^{\myarrow[10] \alpha \neq 0} \lim_{c \rightarrow +\infty} \frac{1}{\alpha} \ (e^{\alpha c}-1)
		\\
		&= \begin{cases}
			-\frac{1}{\alpha} & \text{se } \alpha <0
			\\
			+\infty & \text{se } \alpha >0.
		\end{cases}
	\end{align*}
\end{example}
\end{exbar}

\begin{attbar}
	\begin{equation*}
		\int_{0}^{+\infty} e^{\alpha x} \ \mathrm{d}x \text{ converge } \iff \alpha < 0
	\end{equation*}
	
	Analogamente
	\begin{equation*}
		\int_{-\infty}^{0} e^{\alpha x} \ \mathrm{d}x \text{ converge } \iff  \alpha > 0
	\end{equation*}
\end{attbar}

Dato $ a>0$, scrivendo $e^{\alpha x} = e^{(\alpha \ln a)x}$ si studia la convergenza degli integrali
\begin{equation*}
	\int_{-\infty}^{0} e^{\alpha x} \ \mathrm{d}x \qquad \int_{0}^{+\infty}  e^{\alpha x} \ \mathrm{d}x.
\end{equation*}

(Utile esercizio per casa)


\textbf{Nota Bene}

$f:[1,+\infty[ \ \rightarrow \mathbb{R}$ continua. E' vero o falso che $\lim_{x \rightarrow +\infty} f(x)=0 \Rightarrow \int_{1}^{+\infty} f(x) \ \mathrm{d}x$ converge?

\begin{center}
	\textbf{È falso!!!} Si veda $f(x)=\frac{1}{x}$
\end{center}

E' vero o falso che
$\int_{1}^{+\infty}f(x) \ \mathrm{d}x$ converge $\Rightarrow \lim_{x \rightarrow +\infty} f(x) = 0$?

\begin{center}
	\textbf{È falso!!!}
\end{center}

\begin{center}
	\includegraphics[width=0.75\linewidth]{integrali_impropri/pag85}
	\label{fig:pag85}
\end{center}

Ad ogni numero naturale $n$ costruite un triangolo di base $\frac{1}{2^n}$ e altezza $n$
\begin{equation*}
	\int_{1}^{+\infty} f(x) \ \mathrm{d}x = \sum (\text{aree dei triangoli}) = \sum_{n=1}^{\infty} \frac{1}{2} \frac{1}{2^n} \cdot n = \frac{1}{2} \sum_{n=1}^{\infty} \frac{n}{2^n}, 
\end{equation*}

che è una serie convergente.


\begin{attbar}
	Se $f:[a,b] \rightarrow \mathbb{R}$ è Riemann integrabile, allora è integrabile anche in senso improprio e i due integrali coincidono.
\end{attbar}


\begin{definition}
	$f:[a,b] \ \rightarrow \mathbb{R}, \; a,b \in \mathbb{R} \cup \{\pm \infty\}$, Riemann integrabile in $[c,d] \ \forall \ a < c < d < b$. $f$ si dice integrabile in senso improprio in $ ]a,b[$ se, fissato $\xi \in \ ]a,b[$, lo è in $]a, \xi]$ e in $[\xi,b[$ $\bigg($esistono finiti $\lim_{c \rightarrow a^+} \int_{c}^{\xi} f(x) \ \mathrm{d}x$ e $\lim_{d \rightarrow b^-} \int_{\xi}^{d} f(x) \ \mathrm{d}x \bigg)$ e in tal caso si pone 
	\begin{equation*}
		\int_{a}^{b} f(x)dx = \int_{a}^{\xi} f(x)dx + \int_{\xi}^{b} f(x) dx
	\end{equation*}
	
	Si verifica che la definizione non dipende da punto $\xi$ fissato.
\end{definition}


\begin{exbar}
\begin{example}
	Come si integra $\int_{-\infty}^{+\infty} \frac{1}{1+x^2} \ \mathrm{d}x$?
	
	$x \mapsto \frac{1}{1+x^2}$ è integrabile secondo Riemann in $[c,d] \ \forall  \ c < d$. 
	
	Sia $\xi = 2$. Dobbiamo verificare se la funzione è integrabile in senso improprio in $]-\infty,2] $ e in $ [2, +\infty[$
	\begin{gather*}
		\int_{-\infty}^{2} \frac{1}{1+x^2} \ \mathrm{d}x = \lim_{c \rightarrow -\infty} \int_{c}^{2} \frac{1}{1+x^2} \ \mathrm{d}x = \lim_{c \rightarrow -\infty} \left( \arctan{2 + \frac{\pi}{2}} \right)= \arctan{2 + \frac{\pi}{2}}
		\\
		\int_{2}^{+\infty} \frac{1}{1 + x^2} \ \mathrm{d}x = \lim_{c \rightarrow +\infty} \int_{2}^{c} \frac{1}{1 + x^2} \mathrm{d}x = \lim_{c \rightarrow +\infty} (\arctan{c} - \arctan{2}) = \frac{\pi}{2} - \arctan{2}
		\\
		\Rightarrow \int_{-\infty}^{+\infty} \frac{1}{1+x^2} \ \mathrm{d}x \text{ converge e } \int_{-\infty}^{+\infty} \frac{1}{1+x^2} \ \mathrm{d}x = \int_{-\infty}^{2} \frac{1}{1+x^2} \ \mathrm{d}x + \int_{2}^{+\infty} \frac{1}{1+x^2} \ \mathrm{d}x
	\end{gather*}
\end{example}
\end{exbar}


\begin{exbar}
\begin{example}
	\begin{gather*}
		f(x)=\frac{1}{x^\alpha}, \qquad x \in \ ]0,+\infty[
		\\
		\alpha > 0 \Rightarrow \lim_{x \rightarrow 0^+} f(x)=+\infty.
	\end{gather*}
	
	Preso $\xi=1$, $f$ è integrabile in senso improprio in $]0,+\infty[ \ \iff$ entrambi gli integrali $\int_{0}^{1} \frac{1}{x^\alpha} \ \mathrm{d}x$ e $\int_{1}^{+\infty} \frac{1}{x^\alpha} \ \mathrm{d}x$ convergono. Ma
	\begin{gather*}
		\int_{0}^{1} \frac{1}{x^\alpha} \ \mathrm{d}x \text{ converge} \iff \alpha < 1 
		\\
		\int_{1}^{+\infty} \frac{1}{x^\alpha} \ \mathrm{d}x \text{ converge} \iff \alpha > 1
		\\
		\Rightarrow \int_{0}^{+\infty} \frac{1}{x^\alpha} \ \mathrm{d}x \text{ non converge per alcun valore di } \alpha
	\end{gather*}
\end{example}
\end{exbar}


\begin{exbar}
\begin{example}
	\begin{equation*}
		f(x)= 
		\begin{cases}
			e^x & \text{se } x \leq 1 
			\\
			\frac{1}{x \sqrt{x-1}} & \text{se } x>1
		\end{cases}
	\end{equation*}
	
	vogliamo stabilire se $\int_{-\infty}^{+\infty} f(x) \ \mathrm{d}x$ converge.
	\begin{center}
		\includegraphics[width=0.75\linewidth]{integrali_impropri/pag89}
		\label{fig:pag89}
	\end{center}
	
	Devo studiare $\int_{-\infty}^{1} f(x) \ \mathrm{d}x = \int_{-\infty}^{1} e^x \ \mathrm{d}x$ e, prendendo $\xi=3$,  $\int_{1}^{3} f(x) \ \mathrm{d}x$ e $\int_{3}^{+\infty} f(x) \ \mathrm{d}x$. Se tutti convergono, anche $\int_{-\infty}^{+\infty} f(x) \ \mathrm{d}x$ converge e 
	\begin{equation*}
		\int_{-\infty}^{+\infty} f(x) \ \mathrm{d}x = \int_{-\infty}^{1} f(x) \ \mathrm{d}x + \int_{1}^{3} f(x) \ \mathrm{d}x + \int_{3}^{+\infty} f(x) \ \mathrm{d}x
	\end{equation*}
\end{example}
\end{exbar}

In generale, data $f:]a,b[ \ \rightarrow \mathbb{R}$ integrabile in senso improprio in $]x_0,x_1[, \ ]x_1,x_2[, \ \ldots, \ ]x_{n-1}, \ x_n[$ con $a = x_0 < x_1 < \ldots < x_n = b$, allora $f$ si dice integrabile in senso improprio in $]a,b[$ e si pone 
\begin{equation*}
	\int_{a}^{b} f(x) \ \mathrm{d}x= \sum_{i=1}^{n} \int_{x_{i-1}}^{x_i} f(x) \ \mathrm{d}x.
\end{equation*}

\subsection{Criteri di integrabilità}
\begin{proposition}
	\label{pr: integrabilità definita}
	Sia $f:[a,b[ \ \rightarrow \mathbb{R}, \ b \in \mathbb{R} \cup \{+\infty\}$ tale che $f(x) \geq 0$ definitivamente per $x \rightarrow b^-$ e tale che sia Riemann integrabile in $[a,b] \ \forall c \in [a,b[$. Allora esiste
	\begin{equation*}
		\lim_{c \rightarrow b^-} \int_{a}^{c} f(x) \ \mathrm{d}x
 	\end{equation*}
	cioè l'integrale improprio $\int_{a}^{b} f(x) \ \mathrm{d}x$ è ben definito (converge o diverge a $+\infty$).
\end{proposition}

\begin{dembar}
	\textbf{Dimostrazione} della \textbf{Proposizione \ref{pr: integrabilità definita}}
	
	
	Per semplicità assumiamo $f(x) \geq 0 \ \forall \ x \in [a,b[$. Sia allora $F(c)=\int_{a}^{c} f(x) \ \mathrm{d}x, \ c \in [a,b[$. 
	
	Allora $F$ è monotona crescente, dunque, presi $c_2 > c_1$ si ha
	\begin{equation*}
		F(c_2) - F(c_1) = \int_{a}^{c_2} f(x) \ \mathrm{d}x - \int_{a}^{c_1} f(x) \ \mathrm{d}x = \int_{c_1}^{c_2} f(x) \ \mathrm{d}x \geq 0
	\end{equation*}
	
	perché $f(x) \geq 0 \ \forall x \in [c_1,c_2] \Rightarrow F(c_2) \geq F_{c_1}$.
	
	$F$ ha limite per $c \rightarrow b^-$, cioè esiste
	\begin{equation*}
		\lim_{c \rightarrow b^-} F(c) = \lim_{c \rightarrow b^-} \int_{a}^{b} f(x) \ \mathrm{d}x.
	\end{equation*}
	
	Un risultato analogo si enuncia per funzioni definitivamente $\leq 0$ per $ x \rightarrow b^-$ o per una funzione $f: \ ]a,b] \rightarrow \mathbb{R}, \ a \in \mathbb{R} \cup \{-\infty\}. \ \square$ 
\end{dembar}


Come per le serie ci concentriamo su criteri di integrabilità per funzioni di segno definito.

\subsubsection{Criterio del confronto}
\begin{theorem} (criterio del confronto)
	
\end{theorem}
	\label{th:criterio del confronto integrale}
	$f,g : [a,b[ \ \rightarrow \mathbb{R}, \ b \in \mathbb{R} \cup \{+\infty\}$, Riemann integrabili in $[a,c] \ \forall c \in [a,b[$ e tali che $0 \leq f(x) \leq g(x)$ definitivamente per $x \rightarrow b^-$.
	\begin{enumerate}
		\item Se $\int_{a}^{b} g(x) \ \mathrm{d}x$ converge, allora converge anche $\int_{a}^{b} f(x) \ \mathrm{d}x$.
		\item Se $\int_{a}^{b} f(x) \ \mathrm{d}x$ diverge, allora diverge anche $\int_{a}^{b} g(x) \ \mathrm{d}x$.
	\end{enumerate}
	Un risultato analogo vale per funzioni $f,g: \ ]a,b] \rightarrow \mathbb{R}, \ a \in \mathbb{R} \cup \{-\infty\}$, con ovvie modifiche.


\begin{dembar}
		\textbf{Dimostrazione} del \textbf{Teorema \ref{th:criterio del confronto integrale}}
		
		Per semplicità assumiamo $0 \leq f(x) \leq g(x) \ \forall \ x \in [a,b[$.
		\begin{enumerate}
			\item $\exists$ finito $\lim_{c \rightarrow b^-} \int_{a}^{c} g(x) \ \mathrm{d}x \ \forall \ c \in [a,b[, \ \int_{a}^{c} f(x) \ \mathrm{d}x \leq \int_{a}^{c} g(x) \ \mathrm{d}x$ perché $f(x) \leq g(x)$. Allora 
			\begin{equation*}
				\undercomment{\lim_{c \rightarrow b^-} \int_{a}^{c} f(x) \ \mathrm{d}x} {\text{esiste perché}} {f(x) \geq 0} \leq \lim_{c \rightarrow b^-} \int_{a}^{c} g(x) \ \mathrm{d}x
			\end{equation*}
			
			e quindi $\lim_{c \rightarrow b^-} \int_{a}^{c} f(x) \ \mathrm{d}x $ esiste finito e $\int_{a}^{b} f(x) \ \mathrm{d}x$ è convergente.
			
			\item $\lim_{c \rightarrow b^-} \int_{a}^{c} f(x) \ \mathrm{d}x = +\infty $ ($f(x)\geq 0$)
			\begin{gather*}
				\int_{a}^{c} g(x) \ \mathrm{d}x \geq \uppercomment{\int_{a}^{c} f(x) \ \mathrm{d}x} {\rightarrow +\infty} {\text{per } c \rightarrow + \infty} \ \forall \ c \in [a,b[
				\\
				\Rightarrow \lim_{c \rightarrow b^-} \int_{a}^{c} g(x) \ \mathrm{d}x = +\infty
			\end{gather*}
			 
			per il teorema del confronto sui limiti. $\square$
		\end{enumerate}
\end{dembar}


\subsubsection{Criterio di assoluta integrabilità}
\begin{theorem}
	\label{th:criterio assoluta integrabilità}
	$f:[a,b[ \ \rightarrow \mathbb{R}, \ b \in \mathbb{R} \cup \{+\infty\}$ Riemann integrabile in $[a,c] \ \forall \ c \in [a,b[$ e tale che $|f|$ è integrabile in senso improprio in $[a,b[$ (si dice che l'integrale improprio di $f$ è assolutamente convergente o che converge assolutamente). Allora $\int_{a}^{b} f(x) \ \mathrm{d}x$ converge.
\end{theorem}


\begin{dembar}
	\textbf{Dimostrazione} del \textbf{Teorema \ref{th:criterio assoluta integrabilità}}
	
	\begin{gather*}
		f = f^+ - f^-
		\\
		f^+ = \max \{f(x), 0\} \geq 0 \quad f^- =\min \{f(x), 0\} \geq 0 \quad \forall \ x
		\\
		\left| f(x) \right| = f^+(x) + f^-(x)
	\end{gather*}
	\begin{center}
		\includegraphics[width=0.75\linewidth]{integrali_impropri/pag94(2)}
		\label{fig:pag94}
	\end{center}

	$f^+$ e $f^-$ sono Riemann integrabili  in  $[a,c] \ \forall \ c \in [a,b[$ e inoltre
	\begin{equation*}
		\begin{array}{c}
			0 \leq f^+(x) \leq |f(x)|
			\\
			0 \leq f^-(x) \leq |f(x)|
		\end{array}
		\qquad \forall \ x \in [a,b[
	\end{equation*}
	$\Rightarrow$ per il criterio del confronto $\int_{a}^{b} f^+(x) \ \mathrm{d}x$ e $\int_{a}^{b} f^-(x) \ \mathrm{d}x$ convergono entrambi 
	\begin{equation*}
		\Rightarrow \lim_{c \rightarrow b^-} \int_{a}^{c} f(x) \ \mathrm{d}x = \lim_{c \rightarrow b^-} \left[ \int_{a}^{c}f^+(x) \ \mathrm{d}x - \int_{a}^{c}f^-(x) \ \mathrm{d}x \right]
	\end{equation*}
	esiste finito per il teorema sulla somma dei limiti $\Rightarrow \int_{a}^{b} f(x) \ \mathrm{d}x$ converge. $\square$
\end{dembar}


\begin{attbar}
	Sia $f: [a,b[ \rightarrow \mathbb{R}, \ b \in \mathbb{R} \cup \{+\infty\}$ e se $\int_{a}^{b} f(x) \ \mathrm{d}x$ converge, non è detto che $\int_{a}^{b} |f(x)| \ \mathrm{d}x$ converga.
\end{attbar}


\begin{exbar}
\begin{example}
	\begin{equation*}
		f(x) = \frac{\sin x}{x}, \qquad x \geq 1
	\end{equation*}
	
	e dimostriamo che $ \int_{1}^{+\infty} \frac{\sin x}{x} \ \mathrm{d}x$ converge. Fissiamo $c>1$
	\begin{gather*}
		\int_{1}^{c} \frac{\sin x}{x} \ \mathrm{d}x= -\frac{\cos x}{x} \bigg|_{1}^{c} - \int_{1}^{c} \frac{\cos x}{x^2} \ \mathrm{d}x = \cos 1 - \frac{\cos c}{c} - \int_{1}^{c} \frac{\cos x}{x^2} \ \mathrm{d}x
		\\
		\lim_{c \rightarrow +\infty} \int_{1}^{c} \frac{\sin x}{x} \ \mathrm{d}x = \lim_{c \rightarrow +\infty} \left[\cos 1 - \uppercomment{\frac{\cos c}{c}} {\myarrow[10] 0} {} - \lowercomment{\int_{1}^{c} \frac{cosx}{x^2} \ \mathrm{d}x} {\lim_{c \rightarrow + \infty} \int_{1}^{c} \frac{\cos x}{x^2} \ \mathrm{d}x} {\text{esiste finito}}\right] 
	\end{gather*}

	
	$\frac{|\cos x|}{x^2} \leq \frac{1}{x^2} \ \forall \ x \geq 1$ e $ \int_{1}^{+\infty} \frac{1}{x^2} \ \mathrm{d}x$ converge 
	
	$\Rightarrow \int_{1}^{+\infty} \frac{|\cos x|}{x^2} \ \mathrm{d}x$ converge per il criterio del confronto 
	
	$\Rightarrow \int_{1}^{+\infty} \frac{\cos x}{x^2} \ \mathrm{d}x$ converge per il criterio di assoluta integrabilità
	
	$\Rightarrow \lim_{c \rightarrow +\infty} \int_{1}^{c} \frac{\sin x}{x} \ \mathrm{d}x = \cos 1 - \int_{1}^{+\infty} \frac{\cos x}{x^2} \ \mathrm{d}x$ esiste finito e dunque $\int_{1}^{+\infty} \frac{\sin x}{x} \ \mathrm{d}x$ converge 
	\\[1em]

	Ma $\int_{1}^{+\infty} |\frac{\sin x}{x}|dx =+\infty$.
	
	Presi $k \geq 1, \ k \in \mathbb{N}$
	\begin{gather*}
		k\pi \leq x \leq (k+1)\pi
		\\
		\frac{1}{(k+1)\pi} \leq \frac{1}{x} \leq \frac{1}{k\pi}
		\\
		\int_{k\pi}^{(k+1)\pi} \frac{(\sin x)}{x} \ \mathrm{d}x \geq \int_{k\pi}^{(k+1)\pi} \frac{1}{(k+1)\pi} (\sin x) \ \mathrm{d}x = \frac{2}{(k+1)\pi} 
		\\
		\int_{1}^{+\infty} \frac{|\sin{x}|}{x} \ \mathrm{d}x \geq \sum_{k=1}^{\infty} \int_{k\pi}^{(k+1)\pi} \frac{|\sin{x}|}{x} \ \mathrm{d}x \geq \sum_{k=1}^{\infty} \frac{2}{(k+1)\pi}=+\infty
	\end{gather*}
	
	perché $\frac{2}{(k+1)\pi} \sim \frac{1}{k}$.
\end{example}
\end{exbar}


\subsubsection{Criterio asintotico del confronto}


\begin{comment}	


	
	


	
	

	\paragraph{\textcolor{red}{Teorema}}
	$f,g: [a,b[\rightarrow \R, \,\,\ b\in \R \cup \{+\infty\}$, Riemann integrabili in $[a,c] \,\, \forall c \in [a,b[$ tali che $f(x),g(x)\geq 0$ definitivamente per $x \rightarrow b^-$ e $\lim_{x \rightarrow b^-} \frac{f(x)}{g(x)}=l \in [0,+\infty[$
	\begin{enumerate}
		\item Se $l \in \,\, ]0,+\infty[$, allora $ \int_{a}^{b} f(x) dx$ converge $\Leftrightarrow \int_{a}^{b} g(x) dx$ converge.
		\item Se $l=0$ e $\int_{a}^{}b g(x)dx $ converge, allora $\int_{a}^{b} f(x) dx$ converge.
		\item Se $l=+\infty$ e $\int_{a}^{b} g(x) dx$ diverge, allora $\int_{a}^{b} f(x)dx$ diverge.
	\end{enumerate}
	
	\paragraph{\textcolor{red}{Dimostrazione}}
	\begin{enumerate}
		\item $\lim_{x \rightarrow b^-}\frac{f(x)}{g(x)} = l \in \,\, ]0,+\infty[$ allora $\frac{l}{2}\leq \frac{f(x)}{g(x)}\leq 2l$ definitivamente per $x \rightarrow b^-$ perchè $]\frac{l}{2},2l[$ è un intorno di $l$. Dunque $f(x)\leq 2lg(x)$ e $g(x)\leq \frac{2}{l} f(x)$ definitivamente per $x \rightarrow b^-$. Se $\int_{a}^{b} g(x) dx$ converge $\int_{a}^{b} f(x) dx$ converge e se $\int_{a}^{b} f(x) dx$ converge $\int_{a}^{b} (x) dx$ per il criterio del confronto.
		\item $\lim_{x \rightarrow b^-} \frac{f(x)}{g(x)} = 0 \Rightarrow \frac{f(x)}{g(x)} \leq 1 $ definitivamente per $x\rightarrow b^- \Rightarrow f(x) \leq g(x)$ definitivamente per $x\rightarrow b^- \Rightarrow $ se $\int_{a}^{b} g(x)dx$ converge, allora $\int_{a}^{b} f(x)dx$ converge per il criterio del confronto.
		\item $\lim_{x \rightarrow b^-}\frac{f(x)}{g(x)} = +\infty \Rightarrow \frac{f(x)}{g(x)} \geq 1$ definitivamente per $x \rightarrow b^- \Rightarrow f(x) \geq g(x)$ definitivamente per $x \rightarrow b^- \Rightarrow$ se $\int_{a}^{b} g(x)dx$ diverge, allora $\int_{a}^{b} f(x)dx$ diverge sempre per il criterio del confronto.
	\end{enumerate}
	\begin{flushright}
		\large\Lightning
	\end{flushright}
	
	\paragraph{\textcolor{red}{Esempio importante}}
	Studiamo la convergenza di 
	\begin{equation*}
		\int_{2}^{+\infty} \frac{1}{x^\alpha(\ln x)^\beta}dx, \,\,\,\,\,\, \alpha,\beta \in \R
	\end{equation*}
	\begin{itemize}
		\item Se $\alpha >1 \Rightarrow \int_{2}^{+\infty}\frac{1}{x^\alpha}dx$ converge
		\begin{itemize}
			\item Se $\beta \geq 0 \Rightarrow \frac{1}{x^\alpha(\ln x)^\beta} \leq \frac{1}{x^\alpha}$ per $x>a \Rightarrow \int_{2}^{+\infty} \frac{1}{x^\alpha(\ln x)^\beta}dx$ converge
			\item Se $\beta <0 \Rightarrow \frac{1}{x^\alpha(\ln x)^\beta} = \frac{(\ln x)^{-\beta}}{x^\alpha}$.\\ Preso $\epsilon > 0$ si ha $\ln x \leq x^\epsilon$ definitivamente per $x \rightarrow+\infty$ $\Rightarrow \frac{(\ln x)^{-x}}{x^\alpha} \leq \frac{x^{-\epsilon \beta}}{x^\alpha}=\frac{1}{x^{\alpha+\epsilon\beta}}$ definitivamente per $x\rightarrow +\infty$.\\
			Scelgo $\epsilon$ in modo che $\alpha+\epsilon\beta > 1$, cioè $\epsilon < \frac{1-\alpha}{\beta}>0 \Rightarrow \int_{2}^{+\infty} \frac{1}{x^{\alpha+\epsilon\beta}}dx$ converge $\Rightarrow \int_{2}^{+\infty} \frac{1}{x^\alpha(\ln x)^\beta}dx$ converge.
		\end{itemize}
		\item Se $\alpha < 1 \Rightarrow \int_{2}^{+\infty} \frac{1}{x^\alpha}dx$ diverge
		\begin{itemize}
			\item Se $\beta \geq 0 \Rightarrow \frac{1}{x^\alpha(\ln x)^\beta }\geq$ \textcolor{orange}{?? integrale divergente} $\Rightarrow \frac{1}{x^\alpha(\ln x)^\beta}= \frac{(\ln x)^{-\beta}}{x^\alpha}$.\\
			Fissato $\epsilon>0, \,\,\ln x \leq x^\epsilon $ definitivamente per $x \rightarrow +\infty$ $\Rightarrow (\ln x)^{-\beta}\geq x^{-\epsilon\beta}$ defintivamente per $x \rightarrow +\infty$ $\Rightarrow \frac{1}{x^\alpha (\ln x)^\beta} \geq \frac{x^{-\epsilon\beta}}{x^\alpha} = \frac{1}{x^{\alpha+\epsilon\beta}}$ definitivamente per $x \rightarrow +\infty$.\\
			Scelgo  $\epsilon$ in modo che $ \alpha+\epsilon\beta < 1,\,\,\, \epsilon < \frac{1-\alpha}{\beta} $\\
			$\Rightarrow \int_{2}^{+\infty} \frac{1}{x^{\alpha+\epsilon\beta}}dx$ diverge $\Rightarrow \int_{2}^{+\infty} \frac{1}{x^\alpha(\ln x)^\beta}$ diverge per il criterio del confronto.
			\item  Se $\beta \leq 0 \Rightarrow \frac{1}{x^\alpha(\ln x)^\beta} = \frac{(\ln x)^{-\beta}}{x^\alpha} \geq \frac{1}{x^\alpha}$ e $\int_{2}^{+\infty}\frac{1}{x^\alpha}dx$ diverge $\Rightarrow \int_{2}^{+\infty} \frac{1}{x^\alpha (\ln x)^\beta}dx$ diverge per il criterio del confronto.
		\end{itemize}
		\item Se $\alpha=1 \Rightarrow \int_{2}^{+\infty} \frac{1}{x(\ln x)^\beta} dx$
		\begin{itemize}
			\item Se $\beta=0 \Rightarrow \int_{2}^{+\infty} \frac{1}{x}dx =+\infty$.
			\item Se $\beta \neq 0,1 \Rightarrow \int_{2}^{+\infty} \frac{1}{x(\ln x)^\beta}dx=\lim_{c \rightarrow +\infty} \int_{2}^{c}(\ln x)^{-\beta}\frac{1}{x} dx = \lim_{c \rightarrow +\infty} \frac{1}{1-\beta}(\ln x)^{1-\beta}|_{2}^{c}$ che è uguale a $\frac{(\ln 2)^{1-\beta}}{\beta -1}$ se $\beta >1$ e $+\infty$ se $\beta < 1$.
			\item Se $\beta = 1 \Rightarrow \int_{2}^{+\infty} \frac{1}{x \ln x}dx = \lim_{c \rightarrow +\infty}\int_{2}^{c} \frac{1}{x \ln x}dx = \lim_{c\rightarrow +\infty} \ln(\ln x)|_{2}^{c}=+\infty$.
		\end{itemize}
	\end{itemize}
	\begin{equation*}
		\textcolor{red}{
			\int_{2}^{+\infty} \frac{1}{x^\alpha(\ln x)^\beta}dx
			\,\,\,\, \text{converge} \,\,\,\,
			\Leftrightarrow \alpha > 1 \,\,\, \text{o} \,\,\, (\alpha=1 \,\,\, \text{e}\,\,\, \beta >1)}
	\end{equation*}
	
	\paragraph{\textcolor{red}{Esempio importante}}
	Studiamo la convergenza 
	\begin{empheq}{equation*}
		\int_{0}^{\frac{1}{2}}\frac{1}{x^\alpha|\ln x|^\beta}dx,\,\,\,\,\,\, \alpha,\beta \in \R
	\end{empheq}
	\begin{empheq}{align*}
		\int_{0}^{\frac{1}{2}} \frac{1}{x^\alpha|\ln x|^\beta}dx &= \lim_{c \rightarrow o^+} \int_{c}^{\frac{1}{2}} \frac{1}{x^\alpha|\ln x|^\beta}dx \\
		&= \lim_{c \rightarrow 0^+} \int_{2}^{\frac{1}{c}} \frac{1}{\frac{1}{t^\alpha}|\ln{\frac{1}{t}}|^\beta}\left(\frac{1}{t^2}\right)dt \\
		&= \lim_{c \rightarrow 0^+} \int_{2}^{\frac{1}{c}} \frac{1}{t^{2-\alpha}(\ln t)^\beta}dt \\
		&= \lim_{d \rightarrow +\infty} \int_{2}^{d} \frac{1}{t^{2-\alpha}(\ln 
			t)^\beta} dt =\int_{2}^{+\infty} \frac{1}{t^{2-\alpha}(\ln t)^\beta} dt
	\end{empheq}
	che converge $\Leftrightarrow 2-\alpha>1$ oppure ($2-\alpha = 1$ e $\beta>1$) $\Leftrightarrow \alpha <1$ oppure ($\alpha =1$ e $\beta>1$)\\
	\begin{empheq}{equation*}
		\textcolor{red}{\int_{0}^{\frac{1}{2}}\frac{1}{x^\alpha|\ln x|^\beta}dx \,\,\,\, \text{converge} \,\,\,\,
			\Leftrightarrow \alpha < 1 \,\,\, \text{o} \,\,\, (\alpha=1 \,\,\, \text{e}\,\,\, \beta >1)}
	\end{empheq}
	
	\paragraph{\textcolor{red}{Esempio}}
	Studiare la convergenza dell'integrale improprio
	\begin{empheq}{equation*}
		\int_{-\infty}^{+\infty}e^{-x}dx \,\,\,(=\sqrt{\pi})
	\end{empheq}
	Devo studiare separatamente 
	\begin{empheq}{equation*}
		\int_{-\infty}^{0} e^{-x^2}dx \,\,\,\,\,\, \text{e} \,\,\,\,\,  \int_{0}^{+\infty} e^{-x^2}dx 
	\end{empheq}
	e vedere se entrambi convergono. $x\mapsto e^{-x^2} $ è pari e dunque
	\begin{empheq}{align*}
		\int_{-\infty}^{0} e^{-x^2} dx& = \lim_{c \rightarrow -\infty} \int_{c}^{0} e^{-x^2}dx\\
		&= \lim_{c \rightarrow -\infty} \int_{0}^{-c} e^{-t^2} dt \\
		&= \lim_{d\rightarrow +\infty} \int_{0}^{d} e^{-x^2}dx =\int_{0}^{+\infty} e^{-x^2} dx
	\end{empheq}
	\begin{empheq}{equation*}
		\int_{-\infty}^{0} e^{-x^2}dx \,\,\,\, \text{converge} \,\,\,\, \Leftrightarrow \int_{0}^{+\infty} e^{-x^2}dx \,\,\,\, \text{converge.}
	\end{empheq}
	$0 \leq e^{-x^2}\leq e^{-x} \,\,\,\, \forall x \geq 1$ e $\int_{0}^{+\infty}e^{-x}dx=1$, cioè converge $\Rightarrow$ per il criterio del confronto $\int_{0}^{+\infty} e^{-x^2}dx$ converge.
	
	\paragraph{\textcolor{red}{Esempio}}
	Studiare la convergenza dell'integrale improprio 
	\begin{equation*}
		\int_{1}^{+\infty}\frac{x^\alpha e^{x^4}}{1+e^{4x^4}}dx
	\end{equation*}
	al variare di $\alpha \in \R$.\\
	Per $x\rightarrow+\infty \Rightarrow \frac{x^\alpha e^{x^4}}{1+e^{4x^4}} \sim \frac{x^\alpha e^{x^4}}{x^{4x^4}}=x^\alpha e^{-3x^4}$ quindi $\int_{1}^{+\infty}\frac{x^\alpha e^{x^4}}{1+e^{4x^4}}dx$ converge $\Leftrightarrow \int_{1}^{+\infty} x^\alpha e^{-3x^4}dx$ converge \textcolor{orange}{(Provare per casa ponendo $x=\ln t$)}.\\
	Sicuramente $x^\alpha \leq e^{x^4}$ definitivamente per $x \rightarrow+\infty$ e dunque $x^\alpha e^{-3x^4}\leq e^{-2x^4}\leq e^{-x} $ per $x \geq 1 \Rightarrow \int_{1}^{+\infty} x^\alpha e^{-3x^4}dx $ converge per il criterio del confronto $\forall \alpha \in \R$.
	
	
\end{comment}